\documentclass[12pt]{article}
\begin{document}
\title{Creating a sense of urgency in cover letter}
\author{Lecture by Prof. Charles Duquette. Notes by Alvin Chia}
\date{September 11, 2019}
\maketitle
\setcounter{secnumdepth}{1}
\section{A sense of your limited availability}
In this lesson, you'll learn strategies for creating a sense of urgency
in the hiring manager. You want the hiring manager to feel like he or
she needs to take action immediately, to contact you now before you get
away.

Remember, that in the hiring game, hiring managers are the customer, and
you, or your services, are the product. Like with any other product, if
you have limited availability, you are perceived as more desirable.

This strategy is time honored. A friend of mine used it when selling hot
tubs at trade shows. The first thing he did when setting up a show was
to place brightly colored sold signs on several of the hot tubs. Without
fail, people would ask if they could buy those items. Why? Because
people want what is not available or what has limited availability.

Real estate agents use this trick all the time. They put up bright, red
and white sold signs because they know these encourage more sales in the
neighborhood.

Retail's a big ticket items, know that sold signs encourage sales.
People presume that if the floor models are selling, they must be good
deals. And availability might be limited, they better get one while they
still can.

This is a go-to strategy that experienced marketers are well aware of.
Limited availability creates desire and a sense of urgency. Your cover
letter can do that same with statements such as these.

\begin{quote}
Because we have a limited window of opportunity, please contact me
immediately.

Before I made another decision that makes me unavailable, please contact
me.

Please contact me before {[}date{]} as I have promised to make a
decision regarding a new project by then and may no longer be available.

Please contact me right away as I am in town for only two more weeks.

I am presently between projects and before making a commitment would
like to meet with you.
\end{quote}

These statements do work. They create a sense of urgency and limit your
availability. After all, there's only one of you on the market. Sure,
you have competition. But, you are unique. When you create a perception
of your limited availability, you're using one of the most powerful
motivators known to all businesses.

Think about how many times you've come across a business using phrases
like, for a limited time, expires on, limited availability, first come,
first served, price guaranteed only until sold out.

Businesses use this language because it works.

Remember, your services are the product, the employer is the customer.
Use the powerful principles of marketing that businesses have been using
on all of us for decades.

\section{Use a four letter word to get your way}

There is one four-letter word that has been found to powerfully impact
the effectiveness of an advertisement. It's the word, call.

Split AB tests have repeatedly shown that placing the word, call, in
front of a telephone number increases an ad's response.

The word, call, demands an action. This one word tells anyone who reads
the ad, and has any interest and desire whatsoever in the product,
exactly what action to take. Even though a telephone number would seem
to be enough, including the word, call, somehow adds motivation, for
whatever reason, the word is known to boost responses. So use it.

\section{Cover letter death by cliche}

You do not want to use trite phrases such as this. This sounds like
begging, or the variation, please call me anytime, which sounds
desperate. No one should be this available.

\begin{quote}
Please (do not hesitate to) contact me
\end{quote}

This sounds like begging, or the variation, please call me anytime,
which sounds desperate. No one should be this available

\begin{quote}
You can contact me at \ldots{}
\end{quote}

This sounds condescending. I've actually read cover letters that use,
you have my permission to contact me, which is even worse. I think it's
a fair bet that anyone receiving a cover letter and resume with your
telephone number on it would presume to have permission to call you.

\begin{quote}
If interested, please contact me at \ldots{}
\end{quote}

This is one of the worst. First of all, it's begging. Second, placing
the phrase, if interested, before the call to action, diminishes the
power of the call. The prospect is being asked to question their
interest. It generates feelings of uncertainty. Gee, maybe you´re so
unsure of yourself that you want us to question our own judgement.

\begin{quote}
Contact me at \ldots{} if you have any questions.
\end{quote}

This is limiting. Is this the only reason to call?

\begin{quote}
Looking forward to meeting you.
\end{quote}

This is presumptuous, who´s to say you´re going to meet me, and it´s not
much better if you write, looking forward to hearing from you.

\begin{quote}
Call me at \ldots{} when you have a chance.
\end{quote}

This says, after all this matter's not important, if you have free time
during your busy day, when you never have any free time, go ahead and
pick up the phone.

\textbf{So, what's good?}

\textbf{Try this}

\begin{quote}
I would like to learn more about this position and how I might assist
you. Please call me at \ldots{}
\end{quote}

A direct, simple, straightforward call to action. Combine it with a
limited availability line, and you have a great close.

\section{Limited Availability}

Employers are people. They perceive more value in a job candidate who is
employed or who has limited availability. Whether you're employed or
not, you must demonstrate limited availability to create value in the
mind of the hiring manager

You create that sense of urgency through one of the phrases you learned
about for conveying your limited availability. In the example cover
letter from the last lesson, we used.

\begin{quote}
Please call me at \ldots{} within the next ten days. After that, I must
make a decision about committing to another project and will no longer
be available.
\end{quote}

A sense of urgency is a powerful element of advertising and selling.
Like most any method or idea, a sense of urgency will not work in every
situation. A sense of urgency is most effective when responding to an
open position. Otherwise, not advertising an open position, the
organization will have no urgent need to respond.

\end{document}

